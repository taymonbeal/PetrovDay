\documentclass{article}

% TODO: Make this different for two-sided printing than for two-up printing
% (two-up should have equal L/R margins)
\usepackage[paperwidth=5.5in, paperheight=8.5in, twoside, top=0.8in, bottom=0.8in, inner=0.6in, outer=1.0in]{geometry}

\usepackage{caption}
\usepackage{subcaption}
\usepackage{graphicx}

\begin{document}
\author{James Babcock (jimrandomh@gmail.com)}
\title{Petrov Day}

% Ideas to incorporate {{{
%  X Move the Malthus bit to after three candles are lit (establishing the pattern)
%  X Incorporate a tree of extinct human relatives, in the section about evolution
%    Mention collapsed civilizations: Easter Island, that island hypothesized as being Atlantis, Mayans, etc
%    Work in a mention of Moloch
%    Discuss the post-WW2 measures as a way of ending on a positive note: formation of the UN, twinning, reaties
%  - Mention abandonment of nuclear powered rockets
%    Mention environmental successes
% }}}

% Quotes to maybe use {{{
%
% Through the years her work continued to yield surprising insights, such as
% the unsettling discovery that chimpanzees engage in primitive and brutal
% warfare. In early 1974, a “four-year war” began at Gombe, the first record
% of long-term “warfare” in nonhuman primates. Members of the Kasakela group
% systematically annihilated members of the “Kahama” splinter group. --Jane Goodall

% Most species do their own evolving, making it up as they go along, which is
% the way Nature intended. -–Terry Pratchett

% Most gods throw dice, but Fate plays chess, and you don’t find out til too
% late that he’s been playing with two queens all along. -–Terry Pratchett

% Man still bears in his bodily frame the indelible stamp of his lowly origin. –Charles Darwin
% }}}


\newcommand{\divider}{ %{{{
	% From http://tex.stackexchange.com/questions/32711/totally-sweet-horizontal-rules-in-latex
	\nointerlineskip \vspace{\baselineskip}
	\hspace{\fill}\rule{0.5\linewidth}{.7pt}\hspace{\fill}
	\par\nointerlineskip \vspace{\baselineskip}
} %}}}
\newcommand{\stagedir} [1] { %{{{
	\begin{itshape}
	#1
	\end{itshape}
} %}}}
\newcommand{\blockquote} [2] { %{{{
	\begin{center}
		\parbox{3.5in}{
			``#1''
			\begin{flushright}
				--- #2
			\end{flushright}
		}
	\end{center}
} %}}}
\newcommand{\blockquoteUnattributed} [1] { %{{{
	\begin{center}
		\parbox{3.5in}{
			``#1''
		}
	\end{center}
} %}}}
\newcommand{\blockspacing} [1] { %{{{
	\begin{center}
		\parbox{3.5in}{
			#1
		}
	\end{center}
} %}}}
\newcommand{\blockquoteUnmarked} [2] { %{{{
	\begin{center}
		\parbox{3.5in}{
			#1
			\begin{flushright}
				--- #2
			\end{flushright}
		}
	\end{center}
} %}}}
\newcommand{\poem} [2] { %{{{
	\begin{center}
		\parbox{3.5in}{
			#1
			\begin{flushright}
				--- #2
			\end{flushright}
		}
	\end{center}
} %}}}
\newcommand{\page} [1] { %{{{
	\divider
	#1
	\divider
	\newpage
} %}}}
\newcommand{\pageNoBottomDiv} [1] { %{{{
	\divider
	#1
	\newpage
} %}}}
\newcommand{\sidePage} [1] { %{{{
	\vspace*{0.2in}
	#1
	\newpage
} %}}}
\newcommand{\candelabrum} [1] { %{{{
	\begin{center}
		\includegraphics[width=3in]{#1}
	\end{center}
} %}}}
\newcommand{\candlePassing} { %{{{
	\begin{center}
		\includegraphics[width=2.5in]{images/candlepassing.png}
	\end{center}
} %}}}

% Don't indent paragraphs
\setlength{\parindent}{0cm}

\setlength{\parskip}{\baselineskip}

%%%%%%%%%%%%%%%%%%%%%%%%%%%%%%%%%%%%%%%%%%%%%%%%%%%%%%%%%%%%%%%%%%%%%%

% Front Matter
% Title page {{{
\pagenumbering{gobble}
\sidePage{

\begin{flushright}
\parbox{3in}{
	\begin{center}
		\Huge{Petrov Day}\newline
		\large{September 26}\newline
		\large{By James Babcock}\newline
	\end{center}
}
\end{flushright}

} % }}}
% What-is-this page 1 {{{
\pagenumbering{arabic}
\page{

Petrov Day is a yearly event on September 26 commemorating the anniversary of
the Petrov incident, where a false alarm in the Soviet early warning system
nearly set off a nuclear war. The purpose of the ritual is to make catastrophic
and existential risk emotionally salient, by putting it into historical context
and providing positive and negative examples of how it has been handled. This
is not for the faint of heart and not for the uninitiated; it is aimed at those
who already know what catastrophic and existential risk is, have some
background knowledge of what those risks are, and believe (at least on an
abstract level) that preventing those risks from coming to pass is important.

\divider

You will need:

\begin{itemize} \itemsep0pt \parskip0pt \parsep0pt
	\item A printout of this booklet for each person
	\item A table with enough chairs to seat everyone
	\item A candle holder
	\item 8 candles and a lighter
	\item A fire extinguisher close enough to retrieve if needed
	\item A deck of small index cards or a pad of post-it notes, and some pens
\end{itemize}

} % }}}
% What-is-this page 2 {{{
\page{

Version 1.3.

By James Babcock with content contributions by Ben Landau-Taylor, Adia Porter,
Daniel Speyer and Raymond Arnold, and quotations from many sources. Thanks to
Eliezer Yudkowsky for introducing the idea of commemorating Petrov Day, and to
all the testers, event organizers, and others who've made this possible.

\divider

\stagedir{Stage directions are written in italics, like this. All other text is
to be read aloud. Whenever there is a horizontal line, it becomes the next
person's turn to speak, going clockwise. When reading quotes, you don't need to
read the name and date at the end.}

This day, September 26, is Petrov Day. In 1983, the story of humanity nearly
ended. We're gathered here to remember that moment, and others like it. But to
really feel the magnitude of those events, we need to visit them in their
proper context. Let us begin the story of human history, starting from the
beginning.

} % }}}

% Introduction Section
% Intro page {{{
\page{

%\stagedir{Content warning: Engineered to evoke strong feelings of existential terror.}

\blockquote{In the beginning, the universe was created. This has made a lot of
people very angry, and been widely regarded as a bad move.}{Douglas Adams}

\divider

Let's fast forward over the thirteen billion year long prequel. Our story
begins in the age of myth, of fossils and legends. It starts with the invention
of fire.

} %}}}
% Prometheus quote: Light candle 1 {{{
\page{
\blockquote{I've hunted down and stolen, inside the hollow of a fennel's stalk,
the seed of fire, a gift that has proven itself to be the teacher of every craft
and the greatest resource for humans.  Such is the crime I have committed and
this is the penalty I am to suffer: nailed and chained on this rock beneath the
open sky.}{Prometheus Bound}

\candelabrum{images/candelabrum1.png}

\stagedir{Light the left-most candle, to represent the invention of fire. Point
out the location of the nearest fire extinguisher, then dim or turn off all
other lights in the room.}

} %}}}

% Prehistory

% About Fire {{{
\page {

Fire was first used by Homo Erectus, some time between 400 thousand and 1.7 million years ago. We used fire to make tools and art and to protect us from predatory animals. With fire to keep us warm, we expanded from tropical and subtropical climates to temperate climates with cold winters.

\divider

\blockquote{Most species do their own evolving, making it up as they go along,
which is the way Nature intended. And this is all very natural and organic and
in tune with mysterious cycles of the cosmos, which believes that there's
nothing like millions of years of really frustrating trial and error to give a
species moral fiber and, in some cases, backbone.}{Terry Pratchett}

} %}}}
% Family tree {{{
\sidePage {

\newcommand{\skull} [2] { %{{{
	\parbox{1.3in}{
		\begin{center}
			\includegraphics[width=1.2in]{#1}
			#2
		\end{center}
	}
} %}}}

\skull{images/H_ergaster.png}{Homo Ergaster}
\skull{images/H_erectus.png}{Homo Erectus}
\skull{images/H_florensiensis.png}{Homo Florensiensis}
\skull{images/H_habilis.png}{Homo Habilis}
\skull{images/H_heidelbergensis.png}{Homo Heidelbergensis}
\skull{images/H_neanderthalis.png}{Homo Neanderthalis}
\skull{images/H_rudolfensis.png}{Homo Rudolfensis}


} %}}}
% CUT: Fire and speech {{{
%\page {
%
%Estimates place the invention of fire somewhere from 400,000 to 1.7 million
%years ago, while anatomically modern humans did not appear until 200,000 years
%ago.
%
%Fire is not, by itself, of particularly great importance. It grants protection
%from the cold of winter, and access to foods that would otherwise be inedible;
%but if it had stopped there, then the story of genus homo would have been
%unremarkable. But fire enabled, and favored, creatures with larger brains, and
%some time after that - we don't know when - our ancestors began to speak.
%
%} % }}}


% The invention of language: Candle 2 {{{
\pageNoBottomDiv{
%\blockquote{It's perfectly obvious that there is some genetic factor that
%distinguishes humans from other animals and that it is language-specific. The
%theory of that genetic component, whatever it turns out to be, is what is
%called universal grammar.}{Noam Chomsky}
% Source: http://www.slate.com/articles/health_and_science/new_scientist/2012/03/noam_chomsky_on_linguistics_and_climate_change_.html

\blockquote{It certainly is not a true instinct, for every language has to be learnt. It
differs, however, widely from all ordinary arts, for man has an instinctive
tendency to speak, as we see in the babble of our young children; whilst no
child has an instinctive tendency to brew, bake, or write.}{Charles Darwin,
Descent of Man (1871)}

} %}}}
% Language: Candle 2 {{{
\sidePage {

\stagedir{Take the second candle, which represents the evolution of language. Pass it all the way around the circle. When you hold the candle, it is your turn to speak. What is your name, and when (what year) is your earliest memory?}

\stagedir{Use the first candle to light the second candle.}

\candelabrum{images/candelabrum2.png}

\divider

}
% }}}


% Agriculture: Candle 3 {{{
\page{

Language is the first key to technology; with it, early humans could accumulate
knowledge, not just in genes, but also in sayings and traditions.

They gave names to people around them. They gave names to species of animals
and plants. They gave names to actions and to places and to strategies. They
called some of these good, and called some of them bad. They learned to share
their knowledge, and they learned to deceive each other. They built families
and communities.

They began the long, slow process of taming the wilderness. Their tribes grew
to cities. What became of them?

\stagedir{Take the second candle, which represents language. Use it to light
the third candle, which represents agriculture.}

\candelabrum{images/candelabrum3.png}

} %}}}
% Uplift {{{
\sidePage {

\stagedir{If you or someone else at the table knows the tune to this song, then
sing; if not, read normally.}

\begin{center}
	\parbox{2in}{
		\begin{large}Uplift\end{large}\newline
		By Andrew Eigel
	}
\end{center}

\blockspacing{
	Hands chip the flint, light the fire, skin the kill\newline
	Feet move the tribe track the herd with a will\newline
	Mankind struggles in the cellar of history\newline
	Time to settle down, time to grow, time to breed\newline
	\newline
	Plow tills the soil, plants the seed, pray for rain\newline
	Scythe reaps the wheat, to the mill, to grind the grain\newline
	Towns and cities spread to empire overnight\newline
	Hands keep building as we chant the ancient rite}

\stagedir{Stop here. Go to the next page without reading or singing the rest of the song.}

\blockspacing{
	Coal heats the steam, push the piston, turns the wheel\newline
	Cogs spin the wool, drives the horses made of steel\newline
	Lightning harnessed does our will and lights the dark\newline
	Keep rising higher, set our goal, hit the mark.\newline
	\newline
	\-\hspace{0.5in}Crawl out of the mud,\newline
	\-\hspace{0.5in}Ongoing but slow,\newline
	\-\hspace{0.5in}For the path that is easy\newline
	\-\hspace{0.5in}Ain't the one that lets us grow!\newline
	\newline
	Light to push the sails, read the data, cities glow\newline
	Hands type the keys, click the mouse, out we go!\newline
	Our voices carry round the world and into space\newline
	Send us out to colonize another place¸\newline
	\newline
	Hands make the tools, build the fire, plant the grain.\newline
	Feet track the herd, build a world, begin again.}


}
% }}}


% The Malthusian Trap {{{
\pageNoBottomDiv {
\blockquote{The power of population is so superior to the power of the earth to
produce subsistence for man, that premature death must in some shape or other
visit the human race. The vices of mankind are active and able ministers of
depopulation.  They are the precursors in the great army of destruction, and
often finish the dreadful work themselves. But should they fail in this war of
extermination, sickly seasons, epidemics, pestilence, and plague advance in
terrific array, and sweep off their thousands and tens of thousands. Should
success be still incomplete, gigantic inevitable famine stalks in the rear, and
with one mighty blow levels the population with the food of the
world.}{Thomas Malthus (1798)}
} %}}}
% The Malthusian Trap {{{
\sidePage {

\stagedir{Take the third candle, which represents agricultural society. Pass it
around the circle. Blow it out and return it to its place in the candelabrum.}

\candelabrum{images/candelabrum3b.png}

\divider


}
% }}}


% Agriculture {{{
\page{

Mankind lived in equilibrium between growth and collapse, knowledge gained and
knowledge forgotten. In that world, stories would last only as long as memory,
monuments only as long as wood. For two hundred thousand years, nothing but
genes survived.

\divider

But that was enough. Though they could not preserve knowledge over generations,
they could preserve domesticated plants and animals. They saved the best, and
little by little, the world got easier. And then a select few humans started
writing, and the equilibrium between learning and forgetting was finally
broken.

Of that age, what memories remain?

%This raised the population density, but the Malthusian limit soon caught up.
%The equilibrium between growth and starvation was not broken. Worse, farming
%brought bad diets, poor health, and numerous social ills.
%
%
%\blockquote{Archaeologists studying the rise of farming have reconstructed a crucial stage
%at which we made the worst mistake in human history. Forced to choose between
%limiting population or trying to increase food production, we chose the latter
%and ended up with starvation, warfare, and tyranny.}{Jared Diamond (1987)}
%
%Agriculture did not lift humanity out of the Malthusian trap; it raised the
%population ceiling up, at the expense of peoples' health and well being.
%
%Despite its problems, farming enabled the growth of civilizations, and the
%rulers of some of these civilizations came to realize that they did not want to
%be forgotten. Some built monuments. About 5,000 years ago, some began to write.

}
%}}}
% Invention of writing: Candle 3 again; write names of family members {{{
\page{

\poem{
	I met a traveller from an antique land\newline
	Who said: Two vast and trunkless legs of stone\newline
	Stand in the desert. Near them, on the sand,\newline
	Half sunk, a shattered visage lies, whose frown,\newline
	And wrinkled lip, and sneer of cold command,\newline
	Tell that its sculptor well those passions read\newline
	Which yet survive, stamped on these lifeless things,\newline
	The hand that mocked them and the heart that fed:\newline
	And on the pedestal these words appear:\newline
	``My name is Ozymandias, king of kings:\newline
	Look on my works, ye Mighty, and despair!''\newline
	Nothing beside remains. Round the decay\newline
	Of that colossal wreck, boundless and bare\newline
	The lone and level sands stretch far away}{Percy Bysshe Shelley (1818)}

}

\vspace{0.4in}

\stagedir{Pass the candle once all the way around the circle. When you hold the candle, it is your turn to speak. What is the name of the oldest family member, living or dead, that you can identify?}

\candlePassing

\stagedir{Using the second candle, which represents language, relight the third candle to represent the invention of writing.

\candelabrum{images/candelabrum3.png}

When everyone has written something, continue to the next page.

}
% }}}

% History

% Writing and The Rosetta Stone {{{
\page{

We know more about what the world was like after people started writing, but
not very much survived. One of the most important writings was discovered by
French soldiers in the wall of Fort Julien: the Rosetta Stone, important
because it was written in three languages, two previously untranslatable. After
a long string of honorifics and decrees about taxes and succession, it
declares: there shall be a new holiday!

\divider

\blockquote{On these days in every month, on which there shall be sacrifices
and libations and all the ceremonies customary at the other festivals, and the
offerings shall be given to the priests who serve in the temples. And a festival
shall be kept for King Ptolemy, the Ever-Living, the Beloved of Ptah, the God
Epiphanes Eucharistos, yearly in the temples throughout the land from the 1st
of Thoth for five days ... This decree shall be inscribed on a stela of hard
stone in hieroglyphic and demotic and Greek characters and set up in each of the
first, second, and third temples beside the image of the ever living
king.}{The Rosetta Stone (ca. 196 BC)}

} %}}}
% The Rosetta Stone image {{{
\sidePage{

\includegraphics[width=4in]{images/RosettaStone.png}

}
% }}}


% Scientific Method: Candle 4; write something surprising you learned {{{
\pageNoBottomDiv{

The majority of writing consisted of genealogies, legal codes, and fantastic
stories. But some writing represented progress in philosophy and mathematics,
eventually culminating in the invention of the scientific method.

\divider

\blockquote{Mathematics is the gate and key of the sciences... Neglect of
mathematics works injury to all knowledge, since he who is ignorant of it
cannot know the other sciences or the things of this world. And what is worse,
men who are thus Ignorant are unable to perceive their own ignorance and so do
not seek a remedy.}{Roger Bacon, Opus Majus (1266)}

} %}}}
% Scientific method {{{
\sidePage{

\stagedir{Pass the fourth candle, which represents the scientific method, once all the way around the circle. When you hold the candle, it is your turn to speak. What is something surprising you learned about the world?}

\candlePassing

\stagedir{Using the third candle, which represents writing, light the fourth candle.}

\candelabrum{images/candelabrum4.png}

}
% }}}


% Black Death: Blow out candle 4 {{{
\page{

The scientific method, combined with writing and a university system, marked
the start of an accumulation of knowledge. This could have marked the beginning
of a slow transition into the modern era. Instead, 81 years after Roger Bacon,
history was derailed by a great plague.

\divider

\stagedir{Take the fourth candle, which represents the progress of science.
Hold it, while you read the quote.}

\blockquote{The seventh year after it began, it came to England and first began
in the towns and ports joining on the seacoasts, in Dorsetshire, where, as in
other counties, it made the country quite void of inhabitants so that there
were almost none left alive. ... But at length it came to Gloucester, yea even
to Oxford and to London, and finally it spread over all England and so wasted
the people that scarce the tenth person of any sort was left alive.}{Geoffrey
the Baker, Chronicon Angliae (1360)}

\stagedir{Blow out the candle. Then return it to its place on the candelabrum.}

\candelabrum{images/candelabrum4b.png}

} %}}}
% Black Death cont'd {{{
\page{

The plague killed about half the population of Europe during a four-year
period, and it recurred repeatedly throughout the next three centuries killing
double-digit percentages of the population each time. Between plagues, wars,
and famines, there was little time to build or preserve knowledge.

} %}}}


% Printing press: Re-light candle 4 {{{
\page{

Preserving knowledge required redundancy. In 1439, during the European
Renaissance, Gutenberg perfected a device to do just that.

\divider

Preserving knowledge required redundancy. In 1439, during the European
Renaissance, Gutenberg perfected a device to do just that.

\divider

\blockquoteUnmarked{``Pray, friend Martin, how many impressions can be made by
this press in a day?'' ``About three hundred, if we work it constantly.'' ``Is
it possible!'' exclaimed Peter. ``Now indeed will books multiply. What will the
plodding copyists say to this?''}{Emily Clemens Pearson, Gutenberg and the Art
of Printing (1870)}

\stagedir{Take the fourth candle, which represents the progress of science.

Touch it to each of the other three candles in turn, until it is lit. Then
return it to its place on the candelabrum.}

\candelabrum{images/candelabrum4.png}

} %}}}

% Age of Progress

% Galileo 1610, Halley/Newton 1687{{{
\page{

\blockquote{By the aid of a telescope any one may behold this in a manner which so
distinctly appeals to the senses that all the disputes which have tormented
philosophers through so many ages are exploded at once by the indisputable
evidence of our eyes, and we are freed from wordy disputes upon this subject,
for the Galaxy is nothing else but a mass of innumerable stars planted together
in clusters.}{Galileo, The Starry Messenger (1610)}
% Footnote: Some words have been modernized. Sidereal->Starry, irrefragable->indisputable.

\divider

\blockquote{Matters that vexed the minds of ancient seers,\newline
And for our learned doctors often led\newline
to loud and vain contention, now are seen\newline
In reason's light, the clouds of ignorance\newline
Dispelled at last by science. Those on whom\newline
Delusion cast its gloomy pall of doubt,\newline
Upborne now on the wings that genius lends,\newline
May penetrate the mansions of the gods\newline
And scale the heights of heaven. O mortal men,\newline
Arise! And, casting off your earthly cares,\newline
Learn ye the potency of heaven-born mind,\newline
Its thought and life far from the herd withdrawn!}{Edmund Halley, preface to Newton's Principia Mathematica (1687)}

} %}}}


% Bayes 1763, Watt 1765 {{{
\page{
\blockquote{By calculations similar to these may be determined universally, what
expectations are warranted by any experiments, according to the different number
of times in which they have succeeded and failed; or what should be thought of
the probability that any particular cause in nature, with which we have any
acquaintance, will or will not, in any single trial, produce an effect that has
been conjoined with it.}{Rev. Thomas Bayes, An Essay towards solving a Problem
in the Doctrine of Chances (1763)}

\divider

\blockquote{I was thinking upon the engine at the time, and had gone as far as
the herd's house, when the idea came into my mind that as steam was an elastic
body it would rush into a vacuum, and if a communication were made between the
cylinder and an exhausted vessel it would rush into it, and might be there
condensed without cooling the cylinder. I then saw that I must get rid of the
condensed steam and injection-water if I used a jet as in Newcomen's engine.
Two ways of doing this occurred to me. ... I had not walked farther than the
golf-house when the whole thing was arranged in my mind.}{James Watt (1765)}

} %}}}

% Mendeleev 1864, Bell 1876, candle 5 {{{
\page{

\blockquote{I saw in a dream a table where all elements fell into place as
required. Awakening, I immediately wrote it down on a piece of paper, only in
one place did a correction later seem necessary.}{Dmitri Mendeleev (1864)}

\divider

\blockquote{I then shouted into the mouthpiece the following sentence: Mr.
Watson, Come here, I want to see you. To my delight he came and declared that
he had heard and understood what I said. I asked him to repeat the words. He
answered, ``You said, Mr. Watson come here I want to see you.''}{Alexander Graham Bell (1876)}

\divider

\blockquote{I speak without exaggeration when I say that I have constructed
3,000 different theories in connection with the electric light, each one of
them reasonable and apparently likely to be true. Yet only in two cases did my
experiments prove the truth of my theory. My chief difficulty was in
constructing the carbon filament. ... Every quarter of the globe was ransacked
by my agents, and all sorts of the queerest materials used, until finally the
shred of bamboo, now utilized by us, was settled upon.}{Thomas Edison (1890)}

} %}}}


% The urn of inventions {{{
\page{

\stagedir{Pass the fifth candle, which represents industrialization, around the circle. When you hold the candle, it is your turn to speak. What is a piece of technology that you are grateful for?}

\candlePassing

\stagedir{Using the fourth candle, which represents science, light the fifth candle to represent industrialization.}

\candelabrum{images/candelabrum5.png}

}

\divider

Understanding the world gave us the power to change it.

In 1712, Thomas Newcomen invented the first commercially successful steam engine. It was the first significant power source other than wind, water, and life. In 1769, James Watt designed a more efficient steam engine, paving the way for its use in trains, steamboats, and factories. The Industrial Revolution began.

\blockquote{Modern economic growth is the increase of income per head by a factor of 15 or 20 since the 18th century in places like Britain---and a factor of 8.5 worldwide even including the places that have not had the luck or skill to let it happen fully. It is certainly the most important event in the history of humanity since the domestication of animals and plants, perhaps the most important since the invention of language.}{Deirdre McCloskey (2004)}

\divider

\blockquote{If we continually sample from the urn of possible technological
discoveries before implementing effective means of global coordination,
surveillance, and/or restriction of potentially hazardous information, then we
risk eventually drawing a black ball: an easy-to-make intervention that causes
extremely widespread harm and against which effective defense is
infeasible}{Nick Bostrom (2013)}

\divider

% World War 2 {{{
\page{

%\blockquote{In those days the following happened almost always: I presented
%myself before an assembly of men who believed the opposite of what I wished to
%say and who wanted the opposite of what I believed in. Then I had to spend a
%couple of hours in persuading two or three thousand people to give up the
%opinions they had first held, in destroying the foundations of their views
%with one blow after another and finally in leading them over to take their
%stand on the grounds of our own convictions and our philosophy of life.
%
%I learned something that was important at that time, namely, to snatch from
%the hands of the enemy the weapons which he was using in his reply. I soon
%noticed that our adversaries, especially in the persons of those who led the
%discussion against us, were furnished with a definite repertoire of arguments
%out of which they took points against our claims which were being constantly
%repeated. The uniform character of this mode of procedure pointed to a
%systematic and unified training. And so we were able to recognize the
%incredible way in which the enemy's propagandists had been disciplined, and I
%am proud today that I discovered a means not only of making this propaganda
%ineffective but of beating the artificers of it at their own work. Two years
%later I was master of that art.}{Adolf Hitler, Mein Kampf (1926)}
%
%\blockquote{The art of propaganda lies in understanding the emotional ideas of
%the great masses and finding, through a psychologically correct form, the way
%to the attention and thence to the heart of the broad masses. ... There was no
%end to what could be learned from the enemy by a man who kept his eyes open,
%refused to let his perceptions be ossified, and for four and a half years
%privately turned the stormflood of enemy propaganda over in his brain.}{Adolf Hitler, Mein Kampf (1926)}

\divider

Starting in 1939 and continuing until 1945, World War II killed about 60 million people. Seventeen million people died in the Holocaust, including two thirds of the world’s Jews. The Japanese government perpetrated the Nanking Massacre, the Bataan Death March, the Manila Massacre, and many other atrocities.

\blockquote{The trouble with Eichmann was precisely that so many were like him, and that the many were neither perverted nor sadistic, that they were, and still are, terribly and terrifyingly normal. From the viewpoint of our legal institutions and of our moral standards of judgment, this normality was much more terrifying than all the atrocities put together.}{Hannah Arendt}

And so the world's greatest minds believed they had no choice. They had to
gather in secret, and create the atomic bomb - a weapon to destroy cities, or
the whole world.

} %}}}
% Manhattan Project: Take candle 5 {{{
\page{

\blockquote{We knew the world would not be the same. A few people laughed, a few people cried. Most people were silent. I remembered the line from the Hindu scripture, the Bhagavad-Gita; Vishnu is trying to persuade the Prince that he should do his duty, and to impress him, takes on his multi-armed form and says, 'Now I am become Death, the destroyer of worlds.' I suppose we all thought that, one way or another.}{J. Robert Oppenheimer (1965)}

\divider

\blockquote{I shall write peace upon your wings, and you shall fly around the world so that children will no longer have to die this way.}{Sadako Sasaki, victim of the bombing at Hiroshima}

\stagedir{Pause for a moment of silence.}

} % }}}


% Global coordination
\page{
In 1962, the cold war between the United States and the Soviet Union reached a
crisis. US destroyers under orders to enforce a naval quarantine off Cuba did
not know that the submarines the Soviets had sent to protect their ships were
carrying nuclear weapons. So the Americans began firing depth charges to force
the submarines to the surface, a move the Soviets on board interpreted as the
start of World War III.

\divider

\blockquote{[Savitsky, a submarine captain,] summoned the officer who was assigned to the nuclear torpedo, and ordered him to assemble it to battle readiness. ‘Maybe the war has already started up there, while we are doing summersaults here’ – screamed agitated [Savitsky, justifying his order. ‘We’re gonna blast them now! We will die, but we will sink them all – we will not become the shame of the fleet’. But we did not fire the nuclear torpedo – Savitsky was able to rein in his wrath. After consulting with Second Captain Vasili Alexandrovich Arkhipov and his deputy political officer Ivan Semenovich Maslennikov, he made the decision to come to the surface.}{Vadim Orlov (2007)}

\divider

\blockquote{Moore's Law of Mad Science: Every 18 months, the IQ required to
destroy the world drops by 1 point.}{Source unknown (2005)}

\divider

But the atomic bomb was not the only product of World War II. In 1945, representatives of fifty nations gathered in this very city to create something that had only been created once before: an institution for global coordination on the most serious problems facing humanity.

\divider

\blockquote{We, who have lived through the torture and the tragedy of two world conflicts, must realize the magnitude of the problem before us. We do not need far-sighted vision to understand the trend in recent history. Its significance is all too clear. 

  With ever-increasing brutality and destruction, modern warfare, if unchecked, would ultimately crush all civilization. We still have a choice between the alternatives: the continuation of international chaos--or the establishment of a world organization for the enforcement of peace.}{Harry S Truman, Address to the United Nations Conference in San Francisco}

\divider

\blockquote{All human beings are born free and equal in dignity and rights. They are endowed with reason and conscience and should act towards one another in a spirit of brotherhood. Everyone is entitled to all the rights and freedoms set forth in this Declaration, without distinction of any kind... Furthermore, no distinction shall be made on the basis of the political, jurisdictional or international status of the country or territory to which a person belongs, whether it be independent, trust, non-self-governing or under any other limitation of sovereignty.}{The Universal Declaration of Human Rights}

\divider

\blockquote{We must all hang together or, most assuredly, we shall all hang separately.}{Benjamin Franklin}

\divider

The first attempt at global governance, the League of Nations, ended in the death of millions of innocent people and a mushroom cloud. 

The second attempt, the United Nations, has brought many other organizations in its wake: the International Criminal Court, the World Bank, the World Trade Organization, the World Health Organization. Many believe these institutions are grossly inadequate to face modern challenges that require global cooperation. But the United Nations has not ended in a mushroom cloud.

Not yet.

\divider

\stagedir{Pass the sixth candle, which represents global cooperation, around the circle. When you hold the candle, it is your turn to speak. What is the cause area you currently prioritize?}

\candlePassing

\stagedir{Using the fifth candle, which represents industrialization, light the sixth candle.}

\candelabrum{images/candelabrum6.png}

}

% Petrov incident - Hold the candle close {{{
\page{
We now reach
the historical event that is today's namesake: the Petrov incident. On
September 26, 1983, Stanislav Petrov was the duty officer at the Oko nuclear
early warning system.

\divider

\blockquote{An alarm at the command and control post went off with red lights blinking on
the terminal. It was a nasty shock. Everyone jumped from their seats, looking
at me. What could I do? There was an operations procedure that I had written
myself. We did what we had to do. We checked the operation of all systems - on
30 levels, one after another. Reports kept coming in:  All is correct; the
probability factor is two. ... The highest.}{Stanislav Petrov}

%\stagedir{Take the fifth candle, which represents industry. Hold it over
%the stack of papers until three drops of wax fall.}

\begin{center}
	\includegraphics[width=3.0in]{images/StanislavPetrov.jpg}
\end{center}
}

\divider

\blockquote{I imagined if I'd assume the responsibility for unleashing the third World War
- and I said, no, I wouldn't. ... I always thought of it. Whenever I came on
duty, I always refreshed it in my memory.}{Stanislav Petrov}

% Petrov Incident - Aftermath {{{
\page{

Had he followed procedure, and reported up the chain of command that the Americans had launched missiles, this could have set off a nuclear war. So
instead of telling his superiors what the system was saying, Petrov told his
superiors that it was a false alarm - despite not really knowing this was the
case.

\divider

At the time, he received no award. The incident embarrassed his superiors and
the scientists responsible for the system, so if he had been rewarded, they
would have to be punished. (He received the International Peace Prize thirty
years later, in 2013.)

Things eventually calmed down. The Soviet Union dissolved. Safeguards were put
on most of the bombs, to prevent the risk of accidental (or deliberate but
unauthorized) detonation. 

} %}}}

%%%%%%%%%%%%%%%%%%%%%%%%%%%%%%%%%%%%%%%%%%%%%%%%%%%%%%%%%%%%%%%%%%%%%%%%%%%%%%%

% Ozone layer {{{
\page{
In 1985, Joe Farman, Brian Gardiner, and Jonathan Shanklin made a disturbing
discovery. The ozone layer, the part of our atmosphere that filters out most UV
radiation, was disappearing due to chlorofluorocarbon pollution. Just two years
later a treaty was written to ban the use of CFCs, and two years after that, in
1989, it was in effect. As of today, every country in the United Nations has
ratified the Montreal protocol.

\divider

\blockquote{The hole in the ozone layer is a kind of skywriting. At first it
seemed to spell out our continuing complacency before a witch's brew of deadly
perils. But perhaps it really tells of a newfound talent to work together to
protect the global environment.}{Carl Sagan (1998)}

} % }}}

% AI risk {{{
\page{

But not every threat to humanity is as easy to understand or address as
nuclear weapons or the ozone layer.

\divider

\blockquote{What we do have the power to affect (to what extent depends on how
we define ``we'') is the rate of development of various technologies and
potentially the sequence in which feasible technologies are developed and
implemented. Our focus should be on what I want to call differential
technological development: trying to retard the implementation of dangerous
technologies and accelerate implementation of beneficial technologies,
especially those that ameliorate the hazards posed by other
technologies.}{Nick Bostrom (2002)}

\divider

\blockquote{An unFriendly AI with molecular nanotechnology (or other rapid
infrastructure) need not bother with marching robot armies or blackmail or
subtle economic coercion. The unFriendly AI has the ability to repattern all
matter in the solar system according to its optimization target. This is fatal
for us if the AI does not choose specifically according to the criterion of how
this transformation affects existing patterns such as biology and people. The
AI does not hate you, nor does it love you, but you are made out of atoms which
it can use for something else. The AI runs on a different timescale than you
do; by the time your neurons finish thinking the words ``I should do something''
you have already lost}{Eliezer Yudkowsky, Artificial Intelligence as a Positive
and Negative Factor in Global Risk (2006)}

\divider

\blockquote{The biological threat carries with it the possibility of millions of fatalities and billions of dollars in economic losses. The federal government has acknowledged the seriousness of this threat and provided billions in funding for a wide spectrum of activities across many departments and agencies to meet it. These efforts demonstrate recognition of the problem and a distributed attempt to find solutions. Still, the Nation does not afford the biological threat the same level of attention as it does other threats: There is no centralized leader for biodefense. There is no comprehensive national strategic plan for biodefense. There is no all-inclusive dedicated budget for biodefense… 

The biological threat has not abated. At some point, we will likely be attacked with a biological weapon, and will certainly be subjected to deadly naturally occurring infectious diseases and accidental exposures, for which our response will likely be insufficient.}{Blue Ribbon Study Panel on Biodefense (2004)}

\divider

\blockquote{We might argue whether the probability of nuclear war per year was high or low. But it would make no real difference. If the probability is 10 percent per year, then we expect the holocaust to come in about 10 years. If it is 1 percent per year, then we expect it in about 100 years. 

The lower probability per year changes the time frame until we expect civilization to be destroyed, but it does not change the inevitability of the ruin. In either scenario, nuclear war is 100 percent certain to occur.

This pair of examples brings out a critically important point. Our only survival strategy is to continuously reduce the probability, driving it ever closer to zero. In contrast, our current policies are like repeatedly playing Russian roulette with more and more bullets in the chambers.

We have pulled the trigger in this macabre game more often than is imagined. Each action on our current path has some chance of triggering the final global war. And if we keep pulling the trigger, the gun will inevitably go off. Each "small" war -- in Iran, or Iraq, or Vietnam, or Afghanistan -- is pulling the trigger; each threat of the use of violence -- as in the Cuban missile crisis -- is pulling the trigger; each day that goes by in which a missile or computer can fail is pulling the trigger.

The only way to survive Russian roulette is to stop playing. The only way to survive nuclear roulette is to move beyond war in the same sense that the civilized world has moved beyond human sacrifice and slavery.

When it was merely moral and desirable, it might have been impossible to beat swords into plowshares. Today, it is necessary for survival.}{Hellman (1985)}

}

%%%%%%%%%%%%%%%%%%%%%%%%%%%%%%%%%%%%%%%%%%%%%%%%%%%%%%%%%%%%%%%%%%%%%%%%%%%%%%%

% By the power of these candles... {{{
\page{

\stagedir{Take the first candle. Read the following, then return it.}
\blockspacing{By the power of fire, I am warm and fed and safe.}

\divider

\stagedir{Take the second candle. Read the following, then return it.}
\blockspacing{By the power of language, I can share what I know with others and listen as they share what they know with me.}

\divider

\stagedir{Take the third candle. Read the following, then return it.}
\blockspacing{By the power of writing, I learn the wisdom of the past and pass on my discoveries to the future.}

\stagedir{Take the fourth candle. Read the following, then return it.}
\blockspacing{By the power of science, I see the future, predicting the consequences of actions and events.}

\divider

\stagedir{Take the fifth candle. Read the following, then return it.}
\blockspacing{By the power of industry, I transform the world.}

\divider

\stagedir{Take the sixth candle, and read the following.}
\blockspacing{By the power of coordination, we combine our strengths.}
\stagedir{Pass the sixth candle to the person two seats down from you.}

\divider

We stand on the precipice of destruction. 

For millennia, the human race was at risk of extinction by forces beyond our control, asteroids or cataclysmic climate change. Today, we are safer from these risks than we have ever been. Today, our primary source of danger is from ourselves.

We have the power to end humanity, and perhaps we will.

\stagedir{Take the fifth candle and hold it near the seventh candle, which represents technology that may lead to destruction, close enough that the seventh candle might catch fire but won’t necessarily do so. Then return the fifth candle.}

\divider

And yet there is hope. 

We have eradicated diseases. The war on global poverty is being won. Millions of children will go to bed tonight, happy and healthy, who only a hundred years ago would be dead.

Our power to save comes from the same source as our power to destroy: language and writing, science and industry, coordination and fire. 

Human extinction is a choice. It is up to us to decide whether to make it.

\stagedir{Take the sixth candle and hold it near the eighth candle, which represents the hope of a better world, close enough that the eighth candle might catch fire but won’t necessarily do so. Then return the sixth candle.}

} %}}}
% The End {{{
\page{

The ritual is over.

Your lit candles no longer symbolize anything.

Get up. Stretch.

Warn people before you turn the lights back on.

} %}}}

\page{}

\pagenumbering{gobble}

% Rear cover {{{
\sidePage{}
% }}}

\end{document}
